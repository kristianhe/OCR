\section{CONCLUSION}

% What is the weakest and strongest component of your OCR system (feature engineering, character classification, and character detection)?  Explain your answer.


% What went good/bad with the project?  Any lessons learned?

With the existence of open source and an abundance of machine learning tools, you do not need a degree in computer science to perform machine learning. You should, however, know what type of model to apply to a given problem, and the features are more likely to make the model more capable of discriminating between different classes. This project introduced us to a lot of new models and methods of performing machine learning, which has led to in-depth investigations of documentation and tutorials in order to produce the most successful algorithm. When using existing and well-tested libraries like Keras and Sklearn, a large portion of the effort is to read and understand other's work. For instance, it was helpful to have a basic understanding of how a neural network works, but the process of implementing a CNN in this assignment gave a more profound understanding of how different parts of a neural network can be tweaked to alter the result, based on the area of application.

Much time was used trying to improve the accuracy of the classifiers in Section \ref{sec:class}. With little knowledge of how the training process works, tendencies towards brute force had little effect on the results. Even though we trained and tested multiple generations of models with different hyperparameters, it seemed like most of the model's performance depended on correct preprocessing of the data. Applying and knowing what feature's and how it is fed to the ML model will be of much higher importance the next time we are to set up a pipeline for a machine learning problem.

\printbibliography[title={REFERENCES}]